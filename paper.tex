\documentclass[a4paper]{jpconf}
\usepackage{graphicx}
\usepackage{comment}
\usepackage{cite}
\usepackage{gensymb}



\begin{document}

\title{Validation of an Actuator Line Model Coupled to a Dynamic Stall Model for
Pitching Motions Characteristic to Vertical Axis Turbines} 

\author{Victor Mendoza$^{1}$, Peter Bachant$^{2}$, Martin Wosnik$^{2}$ and Anders Goude$^{1}$ }
\address{$^{1}$ Department of Engineering Sciences, Division of Electricity, Uppsala University, \\Uppsala 751 21, Sweden}
\address{$^{2}$ Center for Ocean Renewable Energy, University of New Hampshire, 24 Colovos Rd.,\\ Durham, NH 03824, USA}
\ead{victor.mendoza@angstrom.uu.se}




\begin{abstract}

    Vertical Axis Turbines (VAWT) have been used for extracting of energy either
    from sea or wind. A simpler design, low cost of maintenance and the
    capability of working independent from the direction of the flow are some of
    the most important advantages over conventional Horizontal Axis Turbines
    (HAWTs). Complex and unsteady fluid dynamics are not easy to model for
    simulation tools. One of the most relevant phenomena is the dynamic stall
    which is inherently to unsteady conditions, this work is focused in its
    study. Dynamic stall is usually used as a passive control for VAWT operating
    conditions, hence the importance of a good understanding. In this study,
    \textbf{a coupled model is implemented with the open-source CFD toolbox
    OpenFOAM for solving the governing equations; an actuator line model is used
    for solving velocity flow field and a dynamic stall model for force
    interactions.} Force coefficients obtained from the model are validated with
    experimental data of pithing airfoil in similar operating conditions as
    H-rotor type.

\end{abstract}



\section{Introduction}

Operating conditions of VAWT are characterized for complex unsteady flows which
give a considerable challenge for both to describe using measurements and to
represent through simulation tools\cite{huyer1996unsteady}. Moreover, VAWT
blades are inherently exposed to cyclic variation in the angle of attack which
gives cyclic blade forces and can give material fatigue damage. Accurate
modeling of the varying forces is therefore very important for the design of
VAWT.

The angle of attack in a fixed pitch VAWT is increasing with decreased tip speed
ration (TSR), and at low TSR (common during stall regulation), the blades will
experience dynamic stall, where the force coefficients for the blade not only
depend on the angle of attack, but also on the rate of change of the angle of
attack. The aim of this study is to investigate the performance of a
Leishman-Beddoes type dynamic stall model when implemented within an actuator
line model, for pitching motion typical to a VAWT.


\section{Methodology}

For solving the governing equations of phenomena involved, a coupled model have
been implemented: the actuator line model is used to calculate the flow
velocities and thereby the angle of attack. The dynamic stall model is used to
calculate blade force coefficients, which the actuator line model need for the
velocity field calculations. In this work it was chosen to validate the model
against wind tunnel data for a pitching blade. This will put the focus on the
force modeling part in the simulation model.

\textbf{Actuator Line Model:} is a three-dimensional and unsteady aerodynamic
model developed by S{\o}rensen and Shen\cite{sorensen1999computation}, used to
study the flow around wind turbines. It is a combination of a solver of
Navier-Stokes equations with a so-called actuator line technique, in which
blades of the turbine are represented by a radial distribution of body forces
along lines. These forces are determined using a dynamic stall model commonly
based on empirical data.

\textbf{Dynamic Stall Modeling:} represented by the Leishman-Beddoes
model\cite{leishman1986generalised} with the modifications of Sheng et
al\cite{sheng2008modified}. It is capable to calculate the unsteady lift,
pitching moment and drag, giving a physical description of the aerodynamics. It
have been widely validated with experimental data\cite{leishman1989semi}.
Separated on three subsystems: an attached flow model for unsteady linear
airloads, a separated flow model for non-linear airloads and a dynamic stall
model for the airloads induced by the leading edge vortex. In this work, the
dynamic stall model described by Dyachuk\cite{dyachuk} has been implemented,
using the library turbinesFoam developed by Bachant\cite{bachant2015simulating}.

NACA0021 airfoil was tested at the Reynolds number of 1000000 (which is a
reasonable value for operating VAWT) during pithing motion similar to the motion
of the VAWT blade. Different amplitude degrees were tried out. In a VAWT with
fixed blades the angle of attack is in function of the TSR of the turbine. A
periodic function is used to represent the variations of the angles of attack
which a VAWT blade experiences analog to the pitching motion,
\begin{eqnarray}
    \alpha = \arctan \left( \frac{\sin \theta}{\lambda + \cos \theta} \right)
\end{eqnarray}
Where $\theta$ is the azimuthal blade angle and $ \lambda $ represents the TSR,
\begin{eqnarray}
    \lambda =  \frac{\Omega R}{V}
\end{eqnarray}
The pitching blade experiments were carried out at Glasgow
University\cite{angell1988collected}\\


\section{Results and discussion}

The reliability of the model results is acceptable, making a good agreement with
experimental results considering the complexity of the flows studied. For a TSR
of 3.34 (Figure \ref{fig1}), the maximum magnitude of the angle of attack is at
17.4\degree which is in the stall region. Both CN and CT curves show the delay
of the flow reattachment, which is characteristic of the dynamic stall
phenomenon. Peaks of simulated values have close agreement with the measured
data.

For a TSR of 2.60, the maximum amplitude of the angle of attack is 22.6, which
is related to a deeper stall region compared to a TSR of 3.34. This is shown
because the ``loop'' of the curve is wider for the lower TSR, and moreover, the
reattachment of the flow is further delayed.

It has been demonstrated how the coupled actuator line model can be coupled to
a modified version of the Leishman-Beddoes model, and the results could be
validated for similar operating conditions of a VAWT.


\begin{figure}[h]
\begin{minipage}{18pc}
\includegraphics[width=18pc]{Fig1.png}
\end{minipage}\hspace{2pc}%
\begin{minipage}{18pc}
\includegraphics[width=18pc]{Fig2.png}
\end{minipage}
\caption{\label{fig1}Normal (left) and tangential (right) force responses during pitching motions of NACA0021 airfoil with a maximum amplitude of 17.4 \degree (analog to a $\lambda$ = 3.34).}
\end{figure}


\begin{figure}[h]
\begin{minipage}{18pc}
\includegraphics[width=18pc]{Fig3.png}
\end{minipage}\hspace{2pc}%
\begin{minipage}{18pc}
\includegraphics[width=18pc]{Fig4.png}
\end{minipage}
\caption{\label{fig2}Normal (left) and tangential (right) force responses during pitching motions of NACA0021 airfoil with a maximum amplitude of 22.6 \degree (analog to a $\lambda$ = 2.60).}
\end{figure}




\section*{References}
%\begin{thebibliography}{9}
%\bibitem{iopartnum} IOP Publishing is to grateful Mark A Caprio, Center for Theoretical Physics, Yale University, for permission to include the {\tt iopart-num} \BibTeX package (version 2.0, December 21, 2006) with  this documentation. Updates and new releases of {\tt iopart-num} can be found on \verb"www.ctan.org" (CTAN).
%\end{thebibliography}

\bibliography{mybib}{}
\bibliographystyle{unsrt}






\end{document}
